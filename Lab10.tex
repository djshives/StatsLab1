% Options for packages loaded elsewhere
\PassOptionsToPackage{unicode}{hyperref}
\PassOptionsToPackage{hyphens}{url}
%
\documentclass[
  english,
  man]{apa6}
\usepackage{amsmath,amssymb}
\usepackage{lmodern}
\usepackage{ifxetex,ifluatex}
\ifnum 0\ifxetex 1\fi\ifluatex 1\fi=0 % if pdftex
  \usepackage[T1]{fontenc}
  \usepackage[utf8]{inputenc}
  \usepackage{textcomp} % provide euro and other symbols
\else % if luatex or xetex
  \usepackage{unicode-math}
  \defaultfontfeatures{Scale=MatchLowercase}
  \defaultfontfeatures[\rmfamily]{Ligatures=TeX,Scale=1}
\fi
% Use upquote if available, for straight quotes in verbatim environments
\IfFileExists{upquote.sty}{\usepackage{upquote}}{}
\IfFileExists{microtype.sty}{% use microtype if available
  \usepackage[]{microtype}
  \UseMicrotypeSet[protrusion]{basicmath} % disable protrusion for tt fonts
}{}
\makeatletter
\@ifundefined{KOMAClassName}{% if non-KOMA class
  \IfFileExists{parskip.sty}{%
    \usepackage{parskip}
  }{% else
    \setlength{\parindent}{0pt}
    \setlength{\parskip}{6pt plus 2pt minus 1pt}}
}{% if KOMA class
  \KOMAoptions{parskip=half}}
\makeatother
\usepackage{xcolor}
\IfFileExists{xurl.sty}{\usepackage{xurl}}{} % add URL line breaks if available
\IfFileExists{bookmark.sty}{\usepackage{bookmark}}{\usepackage{hyperref}}
\hypersetup{
  pdftitle={Lab10 Re-Analysis},
  pdfauthor={Drew Shives1},
  pdflang={en-EN},
  hidelinks,
  pdfcreator={LaTeX via pandoc}}
\urlstyle{same} % disable monospaced font for URLs
\usepackage{graphicx}
\makeatletter
\def\maxwidth{\ifdim\Gin@nat@width>\linewidth\linewidth\else\Gin@nat@width\fi}
\def\maxheight{\ifdim\Gin@nat@height>\textheight\textheight\else\Gin@nat@height\fi}
\makeatother
% Scale images if necessary, so that they will not overflow the page
% margins by default, and it is still possible to overwrite the defaults
% using explicit options in \includegraphics[width, height, ...]{}
\setkeys{Gin}{width=\maxwidth,height=\maxheight,keepaspectratio}
% Set default figure placement to htbp
\makeatletter
\def\fps@figure{htbp}
\makeatother
\setlength{\emergencystretch}{3em} % prevent overfull lines
\providecommand{\tightlist}{%
  \setlength{\itemsep}{0pt}\setlength{\parskip}{0pt}}
\setcounter{secnumdepth}{-\maxdimen} % remove section numbering
% Make \paragraph and \subparagraph free-standing
\ifx\paragraph\undefined\else
  \let\oldparagraph\paragraph
  \renewcommand{\paragraph}[1]{\oldparagraph{#1}\mbox{}}
\fi
\ifx\subparagraph\undefined\else
  \let\oldsubparagraph\subparagraph
  \renewcommand{\subparagraph}[1]{\oldsubparagraph{#1}\mbox{}}
\fi
% Manuscript styling
\usepackage{upgreek}
\captionsetup{font=singlespacing,justification=justified}

% Table formatting
\usepackage{longtable}
\usepackage{lscape}
% \usepackage[counterclockwise]{rotating}   % Landscape page setup for large tables
\usepackage{multirow}		% Table styling
\usepackage{tabularx}		% Control Column width
\usepackage[flushleft]{threeparttable}	% Allows for three part tables with a specified notes section
\usepackage{threeparttablex}            % Lets threeparttable work with longtable

% Create new environments so endfloat can handle them
% \newenvironment{ltable}
%   {\begin{landscape}\centering\begin{threeparttable}}
%   {\end{threeparttable}\end{landscape}}
\newenvironment{lltable}{\begin{landscape}\centering\begin{ThreePartTable}}{\end{ThreePartTable}\end{landscape}}

% Enables adjusting longtable caption width to table width
% Solution found at http://golatex.de/longtable-mit-caption-so-breit-wie-die-tabelle-t15767.html
\makeatletter
\newcommand\LastLTentrywidth{1em}
\newlength\longtablewidth
\setlength{\longtablewidth}{1in}
\newcommand{\getlongtablewidth}{\begingroup \ifcsname LT@\roman{LT@tables}\endcsname \global\longtablewidth=0pt \renewcommand{\LT@entry}[2]{\global\advance\longtablewidth by ##2\relax\gdef\LastLTentrywidth{##2}}\@nameuse{LT@\roman{LT@tables}} \fi \endgroup}

% \setlength{\parindent}{0.5in}
% \setlength{\parskip}{0pt plus 0pt minus 0pt}

% \usepackage{etoolbox}
\makeatletter
\patchcmd{\HyOrg@maketitle}
  {\section{\normalfont\normalsize\abstractname}}
  {\section*{\normalfont\normalsize\abstractname}}
  {}{\typeout{Failed to patch abstract.}}
\patchcmd{\HyOrg@maketitle}
  {\section{\protect\normalfont{\@title}}}
  {\section*{\protect\normalfont{\@title}}}
  {}{\typeout{Failed to patch title.}}
\makeatother
\shorttitle{SHORTTITLE}
\usepackage{csquotes}
\raggedbottom
\ifxetex
  % Load polyglossia as late as possible: uses bidi with RTL langages (e.g. Hebrew, Arabic)
  \usepackage{polyglossia}
  \setmainlanguage[]{english}
\else
  \usepackage[main=english]{babel}
% get rid of language-specific shorthands (see #6817):
\let\LanguageShortHands\languageshorthands
\def\languageshorthands#1{}
\fi
\ifluatex
  \usepackage{selnolig}  % disable illegal ligatures
\fi
\newlength{\cslhangindent}
\setlength{\cslhangindent}{1.5em}
\newlength{\csllabelwidth}
\setlength{\csllabelwidth}{3em}
\newenvironment{CSLReferences}[2] % #1 hanging-ident, #2 entry spacing
 {% don't indent paragraphs
  \setlength{\parindent}{0pt}
  % turn on hanging indent if param 1 is 1
  \ifodd #1 \everypar{\setlength{\hangindent}{\cslhangindent}}\ignorespaces\fi
  % set entry spacing
  \ifnum #2 > 0
  \setlength{\parskip}{#2\baselineskip}
  \fi
 }%
 {}
\usepackage{calc}
\newcommand{\CSLBlock}[1]{#1\hfill\break}
\newcommand{\CSLLeftMargin}[1]{\parbox[t]{\csllabelwidth}{#1}}
\newcommand{\CSLRightInline}[1]{\parbox[t]{\linewidth - \csllabelwidth}{#1}\break}
\newcommand{\CSLIndent}[1]{\hspace{\cslhangindent}#1}

\thispagestyle{headings}

\title{Lab10 Re-Analysis}
\author{Drew Shives\textsuperscript{1}}
\date{}


\affiliation{\vspace{0.5cm}\textsuperscript{1} The Graduate Center, CUNY}

\begin{document}
\maketitle

\pagestyle{headings}

\hypertarget{data-analysis}{%
\subsection{Data analysis}\label{data-analysis}}

We used R {[}Version 4.1.1; (\textbf{R-base?}){]} and the R-packages \emph{dplyr} {[}Version 1.0.7; (\textbf{R-dplyr?}){]}, \emph{forcats} {[}Version 0.5.1; (\textbf{R-forcats?}){]}, \emph{ggplot2} {[}Version 3.3.5; (\textbf{R-ggplot2?}){]}, \emph{papaja} {[}Version 0.1.0.9997; (\textbf{R-papaja?}){]}, \emph{purrr} {[}Version 0.3.4; (\textbf{R-purrr?}){]}, \emph{readr} {[}Version 2.0.2; (\textbf{R-readr?}){]}, \emph{stringr} {[}Version 1.4.0; (\textbf{R-stringr?}){]}, \emph{tibble} {[}Version 3.1.5; (\textbf{R-tibble?}){]}, \emph{tidyr} {[}Version 1.1.4; (\textbf{R-tidyr?}){]}, and \emph{tidyverse} {[}Version 1.3.1; (\textbf{R-tidyverse?}){]} for all our analyses.

\vspace{\baselineskip}

First, I was able to replicate the Fig. 1 from the original paper:

\vspace{\baselineskip}

\begin{tabular}{l|l|r|r}
\hline
Congruency & Posture & mean & SEM\\
\hline
Congruent & Sit & 821.9232 & 16.60384\\
\hline
Congruent & Stand & 807.9599 & 14.93521\\
\hline
Incongruent & Sit & 940.7855 & 17.91041\\
\hline
Incongruent & Stand & 903.9131 & 15.34939\\
\hline
\end{tabular}

\includegraphics{Lab10_files/figure-latex/unnamed-chunk-15-1.pdf}

\vspace{\baselineskip}

Second, I was able to replicate the Stroop effects in both the sitting (\(M_d = -118.86\), 95\% CI \([-133.33\), \(-104.40]\), \(t(49) = -16.52\), \(p < .001\)) and standing (\(M_d = -95.95\), 95\% CI \([-109.41\), \(-82.49]\), \(t(49) = -14.33\), \(p < .001\)) conditions.

The authors found that the Stroop effect shrunk when participants were standing compared to when they were sitting, F(1, 49) = 8.964, p = .004, np2 = .155. I was able to replicate that results using a paired sample t-test, \(M_d = 22.91\), 95\% CI \([7.53\), \(38.29]\), \(t(49) = 2.99\), \(p = .004\).

\vspace{\baselineskip}

Third, I was able to create a power curve analysis for the design:

\includegraphics{Lab10_files/figure-latex/unnamed-chunk-16-1.pdf}

\newpage

\hypertarget{references}{%
\section{References}\label{references}}

\setlength{\parindent}{0in}

Rosenbaum, D., Mama, Y., \& Algom, D. (2017). Stand by Your Stroop: Standing Up Enhances Selective Attention and Cognitive Control. Psychological science, 28(12), 1864-1867.

\begingroup
\setlength{\parindent}{-0.5in}
\setlength{\leftskip}{0.5in}

\hypertarget{refs}{}
\begin{CSLReferences}{0}{0}
\end{CSLReferences}

\endgroup


\end{document}
